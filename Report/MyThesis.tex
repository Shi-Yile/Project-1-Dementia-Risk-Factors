\documentclass[11pt,twoside]{article}
\usepackage{geometry}
\usepackage{enumerate}
\usepackage{latexsym,booktabs}
\usepackage{amsmath,amssymb}
\usepackage{graphicx}
\usepackage{hyperref}
\usepackage[singlespacing]{setspace}
\usepackage{calc}


\geometry{a4paper,left=2cm,right=2.0cm, top=2cm, bottom=2.0cm}

\newtheorem{Definition}{Definition}
\newtheorem{Theorem}{Theorem}
\newtheorem{Lemma}{Lemma}
\newtheorem{Corollary}{Corollary}
\newtheorem{Proposition}{Proposition}
\newtheorem{Algorithm}{Algorithm}
\numberwithin{Theorem}{section}
\numberwithin{Definition}{section}
\numberwithin{Lemma}{section}
\numberwithin{Algorithm}{section}
\numberwithin{equation}{section}

\newcommand{\dottedline}[1]{\makebox[#1]{.\dotfill}}

\begin{document}

\pagestyle{empty}

% =============================================================================
% Title page
% =============================================================================
\begin{titlepage}
\vspace*{.5em}
\center
\textbf{\Large{The School of Mathematics}} \\
\vspace*{1em}
\begin{figure}[!h]
\centering
\includegraphics[width=180pt]{CentredLogoCMYK.jpg}
\end{figure}
\vspace{2em}
\textbf{\Huge{My Incredible Thesis}}\\[2em]
\textbf{\LARGE{by}}\\
\vspace{2em}
\textbf{\LARGE{My Name}}\\
\vspace{6.5em}
\Large{Dissertation Presented for the Degree of\\
MSc in Statistics with Data Science}\\
\vspace{6.5em}
\Large{July 2021}\\
\vspace{3em}
\Large{Supervised by\\Dr Very Important and Dr Strangelove}
\vfill
\end{titlepage}

\cleardoublepage

% =============================================================================
% Executive summary, acknowledgments, and own work declaration
% =============================================================================
\begin{center}
\Large{Executive Summary}
\end{center}

Here comes your executive summary ...

\clearpage

\begin{center}
\Large{Acknowledgments}
\end{center}

Here come your acknowledgments ...

\clearpage

\begin{center}
\Large{University of Edinburgh – Own Work Declaration}
\end{center}


This sheet must be filled in, signed and dated - your work will not be marked unless this is done.
\vspace{1cm}

Name: \dottedline{8cm}

Matriculation Number: \dottedline{6cm}

Title of work: \dottedline{8cm}

\vspace{1cm}

I confirm that all this work is my own except where indicated, and that I have:
\begin{itemize}
\item	Clearly referenced/listed all sources as appropriate	 				
\item	Referenced and put in inverted commas all quoted text (from books, web, etc)	
\item	Given the sources of all pictures, data etc. that are not my own				
\item	Not made any use of the report(s) or essay(s) of any other student(s) either past 	
or present	
\item	Not sought or used the help of any external professional academic agencies for the work
\item	Acknowledged in appropriate places any help that I have received from others	(e.g. fellow students, technicians, statisticians, external sources)
\item	Complied with any other plagiarism criteria specified in the Course handbook
\end{itemize}

I understand that any false claim for this work will be penalised in accordance with
the University regulations	(\url{https://teaching.maths.ed.ac.uk/main/msc-students/msc-programmes/statistics/data-science/assessment/academic-misconduct}).								

\vspace{1cm}

Signature \dottedline{8cm}

\vspace{5mm}

Date \dottedline{8cm}


\clearpage



% =============================================================================
% Table of contents, tables, and pictures (if applicable)
% =============================================================================
\pagestyle{plain}
\setcounter{page}{1}
\pagenumbering{Roman}

\tableofcontents
\clearpage
\listoftables
\listoffigures
\cleardoublepage

\pagenumbering{arabic}
\setcounter{page}{1}

\nocite{*}
\bibliographystyle{abbrv}
\clearpage

\section{Introduction}
\label{sec:intro}
In the introduction, I explain what do I want to know, and why.
Here I will write a very good, precise and brief introduction.
Particularly Section \ref{sec:methods} is good!
\clearpage

\section{Exploratory Data Analysis}  

EasySHARE dataset accessed from Survey of Health, Ageing and Retirement in Europe (SHARE) is used in this report. As a combination of 8 waves of SHARE data, easySHARE contains over 420k observations and 107 variables across a variety of topics, including demographics, household composition, social support and network, childhood conditions, health and health behaviours, functional limitation indices, and work and money. By looking at the summary of each column in the easySHARE dataset, I observed that missing values which are negatively coded exist in most variables and the ratio of missing values in some columns is relatively high in specific waves, e.g. the variable \texttt{maths\_age10} which measures the relative mathematical skills of the respondents at age 10 is only available in wave 3 and 5. Therefore, exploratory data analysis (EDA), including data visualization, variable selection and feature engineering, are very necessary before any further steps. 

\subsection{Cognitive Score}

EasySAHRE dataset does not record diagnosis of dementia (e.g. Alzheimer's disease) in all waves. Instead, it contains the following indices to describe the cognitive functions of respondents:

\begin{itemize}
	\item \texttt{recall\_1}: the number of words recalled in the first trial of the word recall task, ranging from 0 to 10.
	\item \texttt{recall\_2}: the number of words recalled in the delayed word recall task, ranging from 0 to 10.
	\item \texttt{orienti}: orientation of date, month, year and day of week, ranging from 0 (good) to 4 (bad).
	\item \texttt{numeracy\_1}: information on the Mathematical performance, ranging from 1 (bad) to 5 (good).
	\item \texttt{numeracy\_2}: information on the second test on Mathematical performance, ranging from 1 to 5.
\end{itemize}

Therefore, I considered creating a composite cognitive score based on the indices above as a proxy for dementia severity, following Crimmins et al (2011) \cite{Crimmins}. This score would be used as the response variable in subsequent modelling. Before any combination, I visualized the distributions of these columns using histograms:

\begin{figure}[!h]
	\centering
	\includegraphics[width = 0.75\textwidth]{Images/cog_dist.png}
	\caption{Distributions of cognitive function indices}
	\label{fig:cog_dist}
\end{figure}

A high amount of missing values (negative coded) is observed in \texttt{orienti}, \texttt{numeracy\_1} and \texttt{numeracy\_2}. According to the easySHARE guide, the routing filter that only baseline respondents get to the respective questions for \texttt{orienti} and \texttt{numeracy\_1} leads to the large number of missing values in wave 4 to 8. On the other hand, \texttt{numeracy\_2} is only available for respondents of wave 4 to 8 who already participated in one panel waves. Thus, for numeracy scores, only one of both is recorded in most observations. 

\subsubsection{Risk Factors}

Missing values also Since there is over 100 features in the dataset, using all of them to construct the network is inefficient and can lead to biased results. Hence, for model efficiency and accuracy, we need to define appropriate factors and drop redundant variables. Referring to literatures of Crimmins et al (2011), Beam et al (2018) \cite{Beam} and Livingston et al (2020) \cite{Livingston}, I selected modifiable risk factors in 12 different domains. Moreover, for each risk factor, I follow the   

which are \textbf{age},  \textbf{gender}, \textbf{education}, \textbf{alcohol assumption}, \textbf{smoking}, \textbf{obesity}, \textbf{physical activity}, \textbf{depression}, \textbf{social isolation}, \textbf{chronic disease}, \textbf{working status} and \textbf{household finance}. 

Note that there is no single variable indicating diabetes status or hypertension of the respondent in easySHARE dataset. Instead,   


\newpage

\section{Methods}
\label{sec:methods}

In the following, I explain what did I do and how.

I should really cut this short, but BlaBlaBlaBla BlaBlaBlaBlaBla Bla Bla BlaBlaBla Bla Bla BlaBlaBla Bla BlaBla BlaBla Bla BlaBlaBlaBla Bla BlaBla Bla Bla Bla BlaBla BlaBlaBlaBla BlaBlaBlaBlaBla Bla Bla BlaBlaBla Bla. Bla BlaBlaBla Bla BlaBla BlaBla Bla BlaBlaBlaBla Bla BlaBla Bla Bla Bla BlaBla BlaBlaBlaBla. BlaBlaBlaBlaBla Bla Bla BlaBlaBla Bla Bla BlaBlaBla Bla BlaBla BlaBla Bla BlaBlaBlaBla Bla BlaBla Bla Bla Bla BlaBla BlaBlaBlaBla BlaBlaBlaBlaBla Bla Bla BlaBlaBla Bla Bla BlaBlaBla Bla BlaBla BlaBla Bla BlaBlaBlaBla Bla BlaBla Bla Bla Bla BlaBla BlaBlaBlaBla BlaBlaBlaBlaBla Bla Bla BlaBlaBla Bla Bla BlaBlaBla Bla BlaBla BlaBla Bla. BlaBlaBlaBla Bla BlaBla Bla Bla Bla BlaBla BlaBlaBlaBla BlaBlaBlaBlaBla Bla Bla BlaBlaBla Bla Bla BlaBlaBla Bla BlaBla BlaBla Bla BlaBlaBlaBla Bla BlaBla Bla Bla Bla BlaBla.

Note that I start a new paragraph when I have an empty line like this. BlaBlaBlaBla BlaBlaBlaBlaBla Bla Bla BlaBlaBla Bla Bla BlaBlaBla Bla BlaBla BlaBla Bla BlaBlaBlaBla Bla BlaBla Bla Bla Bla. BlaBla BlaBlaBlaBla BlaBlaBlaBlaBla Bla Bla BlaBlaBla Bla Bla BlaBlaBla Bla BlaBla BlaBla Bla BlaBlaBlaBla Bla BlaBla Bla Bla Bla BlaBla BlaBlaBlaBla BlaBlaBlaBlaBla Bla Bla BlaBlaBla Bla Bla BlaBlaBla. Bla BlaBla BlaBla Bla BlaBlaBlaBla Bla BlaBla Bla Bla Bla BlaBla BlaBlaBlaBla BlaBlaBlaBlaBla Bla Bla BlaBlaBla Bla Bla BlaBlaBla Bla. BlaBla BlaBla Bla BlaBlaBlaBla Bla BlaBla Bla Bla Bla BlaBla BlaBlaBlaBla BlaBlaBlaBlaBla Bla Bla BlaBlaBla Bla Bla BlaBlaBla Bla BlaBla BlaBla Bla BlaBlaBlaBla Bla BlaBla Bla Bla Bla BlaBla.\\
But I can also end a line with a double backslash.
\clearpage

\subsection{Models}
\label{sec:Models}

Models are \emph{very} helpful because.
\begin{itemize}
 \item They're good.
 \item They're helpful.
\end{itemize}
\clearpage

\subsection{Techniques}
\label{sec:Techniques}

Techniques even better because.
\begin{enumerate}
 \item They're magnificent.
 \item If they work.
\end{enumerate}
\clearpage

\section{Results}
\label{sec:results}
I this section, I explain what did I discover.

Now it's getting very technical \ldots{} I will cite \cite{shiina,groewe2001}. I will also show my incredible $\alpha$, $\beta$ and $\gamma$ mathematics and do some other fancy stuff.

\subsection{Formulae}

For example look at this
\begin{equation}\label{eqn:aProblem}
\min{}\sum_{s\in\mathcal{S}}Pr_{s}\left[\sum_{t=1}^{T}\left(
\sum_{g\in\mathcal{G}}\left(\alpha_{gts}C_{g}^{0}+
p_{gts}C_{g}^{1}+\left(p_{gts}\right)^{2}C_{g}^{2}\right)
+\sum_{g\in\mathcal{C}}\gamma_{gts}C_{g}^{s}\right)\right],
\end{equation}
and you will see that it has a little number on the side so that I can refer to it as equation (\ref{eqn:aProblem}). Now if I do this
\begin{eqnarray}
\sum_{i=1}^{n}k_{i}&=&20\label{eqn:one}\\
\sum_{j=20}^{m}\delta_{i}&\geq{}&\eta{}\notag
\end{eqnarray}
I can align two formulae and control which one has a number on the side. It is (\ref{eqn:one}). I can also do something like this
\begin{displaymath}
Y_{l}=\left[\begin{array}{cc}
             \left(y_{s}+i\frac{b_{c}}{2}\right)\frac{1}{\tau{}^{2}} &
             -y_{s}\frac{1}{\tau{}e^{-i\theta^{s}}}\\
             -y_{s}\frac{1}{\tau{}e^{i\theta^{s}}} &
             y_{s}+i\frac{b_{c}}{2}
             \end{array}\right],
\end{displaymath}
and it won't have a number on the side. Now if I have to do some huge mathematics I'd better structure it a little and include linebreaks etc. so that it fits on one page.
\begin{eqnarray}\label{eqn:horrible}
p_{l}^{f}&=&G_{l11}\left(2v_{F(l)}\bar{v}_{F(l)}-\bar{v}_{F(l)}^{2}\right)\\
&+&
\bar{v}_{F(l)}\bar{v}_{T(l)}
\left[
B_{l12}\sin{}(\bar{\delta{}}_{F(l)}-\bar{\delta{}}_{T(l)})
+G_{l12}\cos{}(\bar{\delta{}}_{F(l)}-\bar{\delta{}}_{T(l)})
\right]\notag\\
&+&
\left[\begin{array}{r}
      \bar{v}_{T(l)}
      \left[
      B_{l12}\sin{}(\bar{\delta{}}_{F(l)}-\bar{\delta{}}_{T(l)})
      +G_{l12}\cos{}(\bar{\delta{}}_{F(l)}-\bar{\delta{}}_{T(l)})
      \right]\\
      \bar{v}_{F(l)}
      \left[
      B_{l12}\sin{}(\bar{\delta{}}_{F(l)}-\bar{\delta{}}_{T(l)})
      +G_{l12}\cos{}(\bar{\delta{}}_{F(l)}-\bar{\delta{}}_{T(l)})
      \right]\\
      \bar{v}_{F(l)}\bar{v}_{T(l)}
      \left[
      B_{l12}\cos{}(\bar{\delta{}}_{F(l)}-\bar{\delta{}}_{T(l)})
      -G_{l12}\sin{}(\bar{\delta{}}_{F(l)}-\bar{\delta{}}_{T(l)})
      \right]\\
      \bar{v}_{F(l)}\bar{v}_{T(l)}
      \left[
      -B_{l12}\cos{}(\bar{\delta{}}_{F(l)}-\bar{\delta{}}_{T(l)})
      +G_{l12}\sin{}(\bar{\delta{}}_{F(l)}-\bar{\delta{}}_{T(l)})
      \right]\\
      \end{array}\right]
\cdot{}
\left[\begin{array}{c}
      v_{F(l)}-\bar{v}_{F(l)}\\
      v_{T(l)}-\bar{v}_{T(l)}\\
      \delta_{F(l)}-\bar{\delta{}}_{F(l)}\\
      \delta_{T(l)}-\bar{\delta{}}_{T(l)}
      \end{array}\right],\notag
\end{eqnarray}
This is a lot of fun!
\clearpage

\subsection{Important Things}
Finally we should have a nice picture like this one. However, I won't forget that figures and table are environments which float around in my document. So LaTeX will place them wherever it thinks they fit well with the surrounding text. I can try to change that with a float specifier, e.g. [!ht].
%This is a comment. The Compiler ignores it. It is here to remind me that, if I use a .jpeg or .png picture file as below I will need to compile the document with the pdflatex compiler.
\begin{figure}[!ht]
\centering
\includegraphics[width=0.5\textwidth]{scenTree.png}
\caption{Look at this scenario tree with funny times $t_{1}$ and scenarios $s_{1}$ etc.}
\label{fig:scenarioTree}
\end{figure}
Now I want to use one of my own environments. I want to define something.
\begin{Definition}
 I define
$$
\Gamma_{\eta}:=\sum_{i=1}^{n}\sum_{j=i}^{n}\xi{}(i,j)
$$
\end{Definition}
I definitely need some good tables, so I do this.
\begin{table}[!ht]
\centering
\begin{tabular}{|ll|rrrr|}
\hline
Case&Generators&Therm. Units&Lines&Peak load: [MW]&[MVar]\\
\hline\hline
6 bus&3 at 3 buses&2&11&210&210\\
9 bus&3 at 3 buses&3&9&315&115\\
24 bus&33 at 11 buses&26&38&2850&580\\
30 bus&6 at 6 buses&5&41&189.2&107.2\\
39 bus&10 at 10 buses&7&46&6254.2&1387.1\\
57 bus&7 at 7 buses&7&80&1250.8&336.4\\
\hline
\end{tabular}
\caption{Something that doesn't make sense.}
\label{tab:things}
\end{table}
I should really refer to Table \ref{tab:things}.

\subsection{And now something else}

\noindent
Let:
\begin{eqnarray*}
\Omega_0 & = & \{(x,y,z,f): \text{ satisfying } (9)-(19)\}, \\
\Omega_1 & = & \{(x,y,z,f): \text{ satisfying } (9),(11)-(20)\}, \\
\overline{\Omega}_0 & = & \{\textbf{0}\leq (x,y,z,f) \leq \textbf{1}: \text{ satisfying } (9)-(18)\}, \\
\overline{\Omega}_1 & = & \{\textbf{0}\leq (x,y,z,f) \leq \textbf{1}: \text{ satisfying } (9),(11)-(18),(20)\} \,.
\end{eqnarray*}
%
where $\textbf{0}$ and $\textbf{1}$ are vectors of appropriate dimensions with 0's and 1's, respectively.
Next we see that both $\Omega_0$ and $\Omega_1$ give equivalent formulations for the A-MSSP. In particular, the following statements hold:

\begin{Proposition}
$\Omega_0 \subseteq \Omega_1$.
\end{Proposition}

\noindent
\textbf{Proof.}
Let us suppose there exists $(x,y,z,f) \in \Omega_1$ such that $(x,y,z,f) \notin \Omega_0$.
Then, there exist indices $i \in I$ and $t \in \{0,\ldots,|T|-s_i\} $ with $x_i^t > \displaystyle 0.5\,\left( \sum_{h=1}^{s_i} x_i^{t+h} +1\right)$.
By definition, $x_i^t = 1$ and $x_i^{t+h} = 0$ for all $h \in \{1,\dots,s_i\}$. By~(11) and (12), $\displaystyle \sum_{h=1}^{s_i} f_i^{th}=1$, so $f_i^{th'}=1$ for some $h' \in \{1,\dots,s_i\}$.
But then,
\[ 0 \:=\: x_i^{t+h'} \:=\: \sum_{h=\max \{1, t+h'-(|T|-s_i)\}}^{\min\{s_i,t+h'\}} f_i^{t+h'-h,h} \:\ge\: f_i^{th'} \:=\: 1 \,,
\]
as $h' \in [\max \{1, t+h'-(|T|-s_i)\}, \min\{s_i,t+h'\}]$.
\hfill $\square$
\bigskip

\noindent
This immediately gives us
\begin{Corollary}
AS is a valid formulation for the A-MSSP.
\end{Corollary}

\noindent
Next we compare the Linear Programming (LP) relaxations of the two formulations.

\begin{Proposition}
$\overline{\Omega}_1 \subseteq  \overline{\Omega}_0 $.
\end{Proposition}

\noindent
\textbf{Proof.}
Homework
\hfill $\square$
\cleardoublepage


\section{Conclusions}
I have no idea how to conclude, so I don't write much. But the stuff that follows is important.
\clearpage

%the entries have to be in the file literature.bib
\bibliography{literature}
\clearpage

\appendix
\section*{Appendices}
\addcontentsline{toc}{section}{Appendices}

\section{An Appendix}
\label{app:one}

Some stuff.
\clearpage

\section{Another Appendix}
\label{app:two}

Some other stuff.

\end{document}
